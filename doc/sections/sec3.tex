\section{Dependency Score}
Other way to look at similarity is to see how much a set 
includes elments of the other set compare to all the elements
in the set. Let's say we have two documents and one of those is 
subset of the other. if the second document has lots of strings compare
to the first one. This causes low smilarity score. So we introduce another 
score to recognize how much one document depends on other. So the score
is defined as following:

\begin{equation}
\label{eq:inc-s}
  S(A,B) = \frac{|LCS(A,B)|}{|A|} 
\end{equation}
shows the dependency of set A on set B.
\subsection{Partial Dependency}
What if the documents are not the subset of each other nor similar but
still there is some parts in them that are exactly the same. In LCS, we 
don't care about if the similar words occurs in dense or in a sparse
space. As an example consdier two following random texts. The colored parts
shows the parts that are similar. As you can see if the uncolored text increase
the Similarity and Dependency score is going to decrease, although they have
exactly similar parts in them. 

\begin{table}[H]
\centering
\begin{tabular}{p{3in}p{3in}}
  \subsubsection*{Text 1}  
  \textbf{\textcolor{codepurple}{Both rest of know draw fond post as. 
  It agreement defective to excellent. 
  Feebly do engage of narrow. Extensive 
  repulsive belonging depending if promotion 
  be zealously as.}} Preference inquietude ask 
  now are dispatched led appearance. Small meant 
  in so doubt hopes. Me smallness is existence attending
  he enjoyment favourite affection. Delivered is to ye belonging 
  enjoyment preferred. Astonished and acceptance men two discretion. 
  Law education recommend did objection how old. 
  \textbf{\textcolor{codepurple}{To sure calm much most long me mean. Able rent long in do we. 
  Uncommonly no it announcing melancholy an in. Mirth learn it 
  he given. Secure shy favour length all twenty denote. He felicity 
  no an at packages answered opinions juvenile.}}

    &
    \subsubsection*{Text 2} 
   \textbf{\textcolor{codepurple}{Both rest of know draw fond post as. 
  It agreement defective to excellent. 
  Feebly do engage of narrow. Extensive 
  repulsive belonging depending if promotion 
  be zealously as.}}Performed suspicion in certainty 
  so frankness by attention pretended. Newspaper or 
  in tolerably education enjoyment. Extremity excellent 
  certainty discourse sincerity no he so resembled. Joy 
  house worse arise total boy but. Elderly up chicken do at 
  feeling is. Like seen drew no make fond at on rent. Behaviour 
  extremely her explained situation yet september gentleman are who. 
  Is thought or pointed hearing he. 
  \textbf{\textcolor{codepurple}{To sure calm much most long me mean. Able rent long in do we. 
  Uncommonly no it announcing melancholy an in. Mirth learn it 
  he given. Secure shy favour length all twenty denote. He felicity 
  no an at packages answered opinions juvenile.}}
\end{tabular}
\caption{Two random texts. Colored parts are similar}
\label{tab:texts}
\end{table}

For the complexity of recognizing this part
we are not going to define a score, and only use diagrams to gain the insight
about the existance of this. You can see the occurance distribution graph of LCS in
Figure \ref{fig:dist}.
\begin{figure}[H]
  \centering
  \scalebox{.35}{\input{images/occurance.pgf}}
  
  \caption{LCS occurance distribution of texts in Table \ref{tab:texts}}
  \label{fig:dist}
\end{figure}

\subsection{Implementing Dependency Score in Python}
Now let's implement the the Dependency Score using formula in the
Equation \ref{eq:inc-s}. For the partial inclusion, the occurance
indexes of LCS will be found to draw the the graph later in another
part of the program. The code is shown in the Listing \ref{lis:inc}.
\begin{python}[caption=Inclusion Score, label=lis:inc]
def depend_score(t1, t2, lcs):
    '''
    t1: list of words
    t2: list of words
    lcs: longest common sequence of t1 and t2
    return: return dependency rate of t1 on t2'''
    return  len(lcs)/len(t1)


def dist_finder(text, words):
    '''
    text: list of words
    words: List of word occurances in text
    return: index of words occurances in text'''
    counter = 0
    dist_list = []

    for i, word in enumerate(text):
        if len(words) == counter:
            break
        if words[counter] == word:
            dist_list.append(i)
            counter += 1

    return dist_list
\end{python}