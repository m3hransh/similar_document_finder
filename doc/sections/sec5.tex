\section{Find The Most Similar Code in Geeksforgeeks}
In this section, we write an application that will search 
geeksforgeeks on the Google by some specified keywords, and try 
to find the most similar code to the code that is given by the user.
So first, we need to find the URLs by google search and then scrape 
code sections of each URL to compare it with the given document. 
For the comparison and Similarity score we will use the previous 
implementations.


\subsection{Web Scraping}
\textbf{Web scraping} is the porcess of gathering information from the
Internet. The word usually refer to a process that involves automation.
There challenges to web scraping. One of those are the websites varies
in the structure. So we need unique way of treating those and gathering
information we want.
another challenge about web scraping is that through time the structure
website can change and the program you wrote may become irrelevant in the
future.

\subsection{Implementing Web Scraping in Python}
For implementation the searching part we will use \href{https://pypi.org/project/google/}{\textbf{google}} python package.
For scraping part we will use \textbf{request} and \textbf{BeautifulSoup}.
In this section, we only consider, gathering codes from
\href{https://www.geeksforgeeks.org/}{Geeksforgeeks.org} website to not 
deal with the variety structure of websites and can gather exactly the
data we want. So first, with the use of google search we find relevant geeksforgeeks
URLs and then scrape top list URLs to get relevant codes against some keywords
given by the user. The following Listing \ref{lis:scrape} is the implementation of two functions
that do these tasks.

\pythonexternal[caption=Implementing of Web Scraping and Google search in Python,label=lis:scrape]{codes/scraping.py}